\subsection{Intuitionistic Logic}
\label{sec:bg_il}
Intuitionistic logic is also called constructive logic. As a formalization of intuitionism, it differs from classical logic not only in that syntactically some laws in classical logic are not axioms of the system but also in the meaning for statements to be true. The judgments about statements are based on the existence of a proof or a ``construction'' of that statement. This existence property makes it practically useful, e.g. provided that a constructive proof that an object exists, one can turn it into an algorithm for generating an example of the object.

One vertex in the correspondence-triangle is intuitionistic propositional logic. So the introduction to intuitionistic logic in this dissertation is up to the propositional one.

%---------------------------------------------------------------------------%
%Syntax
\subsubsection{Syntax}
\label{sec:bg_il_s}
The language of intuitionistic propositional logic is similar to the one of classical propositional logic. Customarily, people use $ \bot , \to , \land , \lor $ as basic connectives and treat $ \neg \varphi $ as an abbreviation for $ \varphi \to \bot $.

\begin{definition}[\textbf{Formulas}]
\label{definition:formulas}
Given an infinite set of propositional variables, the \emph{set} $ \Phi $ \emph{of formulas} in intuitionistic propositional logic is defined by induction, represented in the following grammar:
\[
\Phi ::= P | \bot | \neg \Phi | ( \Phi \to \Phi ) | ( \Phi \land \Phi ) | ( \Phi \lor \Phi )
\]
where $ P $ is a \emph{propositional variable}, $ \bot $ is \emph{contradiction}, $ \neg $ is \emph{negation}, $ \to $ is \emph{implication}, $ \land $ is \emph{conjunction}, and $ \lor $ is \emph{disjunction}.
\end{definition}

Given a set $ \Gamma $ of propositions and a proposition $ \varphi $, the relation $ \Gamma \vdash \varphi $ says that there is a derivation with conclusion $ \phi $ from hypotheses in $ \Gamma $. Here $ \Gamma $ is also called context. If $ \Gamma $ is empty, we write $ \vdash \varphi $ and say that $ \varphi $ is a theorem.

For notational convenience, we use the following abbreviations:
\begin{myitemize}
\item $ \varphi _1 , \varphi _2 , \cdots , \varphi _n $ for $ \{ \varphi _1 , \varphi _2 , \cdots , \varphi _n \} $
\item $ \Gamma , \varphi $ for $ \Gamma \cup \{ \varphi \} $
\end{myitemize}

The natural deduction system allows one to derive conclusions from premises. The axiom and inference rules in this deduction system show how the relation $ \vdash $ is built.

\begin{definition}[\textbf{Natural Deduction System}]
\label{definition:nat_ded_sys}
Given a set of propositional variable, the relation $ \Gamma \vdash \varphi $ is obtained by using the following axiom and inference rules
\begin{myitemize}
\item \emph{Axiom}
\begin{prooftree}
\AxiomC{}
\RightLabel{($ axiom $)}
\UnaryInfC{$ \varphi \vdash \varphi $}
\end{prooftree}
Since proposition $ \varphi $ appears in the context, one can conclude it from the context.

\item \emph{Adding hypotheses into context}
\begin{prooftree}
\AxiomC{$ \Gamma \vdash \varphi $}
\RightLabel{($ add $)}
\UnaryInfC{$ \Gamma , \psi \vdash \varphi $}
\end{prooftree}
This rule allows one to add additional hypotheses into the context.

\item \emph{$ \to $-introduction}
\begin{prooftree}
\AxiomC{$ \Gamma , \varphi \vdash \psi $}
\RightLabel{($ \to $I)}
\UnaryInfC{$ \Gamma \vdash \varphi \to \psi $}
\end{prooftree}
If one can derive $ \psi $ from the context with $ \phi $ as a hypothesis, then $ \varphi \to \psi $ is derivable from the same context without $ \varphi $.

\item \emph{$ \to $-elimination}
\begin{prooftree}
\AxiomC{$ \Gamma \vdash \varphi \to \psi $}
\AxiomC{$ \Gamma \vdash \varphi $}
\RightLabel{($ \to $E)}
\BinaryInfC{$ \Gamma \vdash \psi $}
\end{prooftree}
If both the conditional claim ``if $ \varphi $ then $ \psi $'' and $ \varphi $ are provided, one can conclude $ \psi $. As mentioned in the beginning, this is a very common inference rule which is also called \emph{modus ponens}.

\item \emph{$ \land $-introduction}
\begin{prooftree}
\AxiomC{$ \Gamma \vdash \varphi $}
\AxiomC{$ \Gamma \vdash \psi $}
\RightLabel{($ \land $I)}
\BinaryInfC{$ \Gamma \vdash \varphi \land \psi $}
\end{prooftree}
If both $ \varphi $ and $ \psi $ are derivable from $ \Gamma $, $ \varphi \land \psi $ is also derivable.

\item \emph{$ \land $-elimination}
\begin{center}
\AxiomC{$ \Gamma \vdash \varphi \land \psi $}
\RightLabel{($ \land $E$ _1 $)}
\UnaryInfC{$ \Gamma \vdash \varphi $}
\DisplayProof \hspace{10pt}
\AxiomC{$ \Gamma \vdash \varphi \land \psi $}
\RightLabel{($ \land $E$ _2 $)}
\UnaryInfC{$ \Gamma \vdash \psi $}
\DisplayProof
\end{center}
Provided that conjunction $ \varphi \land \psi $ is derivable from $ \Gamma $, both of its components are also derivable.

\item \emph{$ \lor $-introduction}
\begin{center}
\AxiomC{$ \Gamma \vdash \varphi $}
\RightLabel{($ \lor $I$ _1 $)}
\UnaryInfC{$ \Gamma \vdash \varphi \lor \psi $}
\DisplayProof \hspace{10pt}
\AxiomC{$ \Gamma \vdash \psi $}
\RightLabel{($ \lor $I$ _2 $)}
\UnaryInfC{$ \Gamma \vdash \varphi \lor \psi $}
\DisplayProof
\end{center}
One can conclude disjunction $ \varphi \lor \psi $ from either $ \varphi $ or $ \psi $.

\item \emph{$ \lor $-elimination}
\begin{prooftree}
\AxiomC{$ \Gamma \vdash \varphi \to \rho $}
\AxiomC{$ \Gamma \vdash \psi \to \rho $}
\AxiomC{$ \Gamma \vdash \varphi \lor \psi $}
\RightLabel{($ \lor $E)}
\TrinaryInfC{$ \Gamma \vdash \rho $}
\end{prooftree}
If $ \rho $ follows $ \varphi $, $ \rho $ follows $ \psi $ and $ \varphi \lor \psi $, one can conclude $ \rho $.

\item \emph{$ \bot $-elimination}
\begin{prooftree}
\AxiomC{$ \Gamma \vdash \bot $}
\RightLabel{($ \bot $E)}
\UnaryInfC{$ \Gamma \vdash \varphi $}
\end{prooftree}
From contradiction $ \bot $, we can derive any propositions. This rule is also called \emph{principle of explosion} or \emph{ex falso quodlibet}.

\end{myitemize}
\end{definition}

What syntactically makes it different from classical propositional logic is that 

%---------------------------------------------------------------------------%
%Proof
\subsubsection{Proofs}
\label{sec:bg_il_p}

Some examples are given here to show how the rules above are used to build a theorem:
\begin{myitemize}
\item[(1)] $ \varphi , \neg \varphi \vdash \bot $
\begin{prooftree}
\AxiomC{}
\UnaryInfC{$ \varphi \vdash \varphi $}
\RightLabel{($ add $)}
\UnaryInfC{$ \varphi , \neg \varphi \vdash \varphi $}
  \AxiomC{}
  \UnaryInfC{$ \neg \varphi \vdash \neg \varphi $}
  \RightLabel{($ add $)}
  \UnaryInfC{$ \varphi , \neg \varphi \vdash \neg \varphi $}
  \RightLabel{($ \neg \varphi $ stands for $ \varphi \to \bot $)}
  \UnaryInfC{$ \varphi , \neg \varphi \vdash \varphi \to \bot $}
\RightLabel{($ \to $E)}
\BinaryInfC{$ \varphi , \neg \varphi \vdash \bot $}
\end{prooftree}
\item[(2)] $ \vdash \varphi \to \neg \neg \varphi $
\begin{prooftree}
\AxiomC{}
\RightLabel{(1)}
\UnaryInfC{$ \varphi , \neg \varphi \vdash \bot $}
\RightLabel{($ \to $I)}
\UnaryInfC{$ \varphi \vdash \neg \varphi \to \bot $}
\RightLabel{($ \neg \neg \varphi $ stands for $ \neg \varphi \to \bot $)}
\UnaryInfC{$ \varphi \vdash \neg \neg \varphi $}
\end{prooftree}
\item[(3)] $ \vdash \varphi \to ( \psi \to \varphi ) $
\begin{prooftree}
\AxiomC{}
\UnaryInfC{$ \varphi \vdash \varphi $}
\RightLabel{($ add $)}
\UnaryInfC{$ \varphi , \psi \vdash \varphi $}
\RightLabel{($ \to $I)}
\UnaryInfC{$ \varphi \vdash \psi \to \varphi $}
\RightLabel{($ \to $I)}
\UnaryInfC{$\vdash \varphi \to ( \psi \to \varphi ) $}
\end{prooftree}
\item[(4)] $ \vdash ( \neg \varphi \lor \psi ) \to ( \varphi \to \psi ) $
\begin{prooftree}
\AxiomC{}
\RightLabel{(1)}
\UnaryInfC{$ \neg \varphi , \varphi \vdash \bot $}
\RightLabel{($ \bot $E)}
\UnaryInfC{$ \neg \varphi , \varphi \vdash \psi $}
\RightLabel{($ \to $I)}
\UnaryInfC{$ \neg \varphi \vdash \varphi \to \psi $}
\RightLabel{($ \to $I)}
\UnaryInfC{$ \vdash \neg \varphi \to ( \varphi \to \psi ) $}
  \AxiomC{}
  \UnaryInfC{$ \psi \vdash \psi $}
  \RightLabel{($ add $)}
  \UnaryInfC{$ \psi , \varphi \vdash \psi $}
  \RightLabel{($ \to $I)}
  \UnaryInfC{$ \psi \vdash \varphi \to \psi $}
  \RightLabel{($ \to $I)}
  \UnaryInfC{$\vdash \psi \to ( \varphi \to \psi ) $}
    \AxiomC{}
    \UnaryInfC{$ \neg \varphi \lor \psi \vdash \neg \varphi \lor \psi $}
\RightLabel{($ \lor $E)}
\TrinaryInfC{$ \neg \varphi \lor \psi \vdash \varphi \to \psi $}
\RightLabel{($ \to $I)}
\UnaryInfC{$ \vdash ( \neg \varphi \lor \psi ) \to ( \varphi \to \psi ) $}
\end{prooftree}
\end{myitemize}
