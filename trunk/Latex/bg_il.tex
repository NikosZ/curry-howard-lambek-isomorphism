\subsection{Intuitionistic Logic}
\label{sec:bg_il}
Intuitionistic logic is also called constructive logic. As a formalisation of intuitionism, it differs from classical logic not only in that some laws in classical logic are not axioms of the system, but also in the meaning for statements to be true. The judgements about statements are based on the existence of a proof or a ``construction'' of that statement. This existence property makes it practically useful, e.g. provided that a constructive proof that an object exists, one can turn it into an algorithm for generating an example of the object.

One vertex in the correspondence-triangle is intuitionistic propositional logic. Consequently, the introduction to intuitionistic logic in this dissertation is up to the propositional one.

%---------------------------------------------------------------------------%
%Syntax
\subsubsection{Syntax}
\label{sec:bg_il_s}
The language of intuitionistic propositional logic is similar to the one of classical propositional logic. Customarily, people use $ \bot , \to , \land , \lor $ as basic connectives and treat $ \neg \varphi $ as an abbreviation for $ \varphi \to \bot $.

\begin{definition}[\textbf{Formulas}]
\label{definition:formulas}
Given an infinite set of propositional variables, the \emph{set} $ \Phi $ \emph{of formulas} in intuitionistic propositional logic is defined by induction, represented in the following grammar:
\begin{center}
$ \Phi \hspace*{6pt} ::= \hspace*{6pt} P \hspace*{6pt} | \hspace*{6pt} \bot \hspace*{6pt} | \hspace*{6pt} \neg \Phi \hspace*{6pt} | \hspace*{6pt} ( \Phi \to \Phi ) \hspace*{6pt} | \hspace*{6pt} ( \Phi \land \Phi ) \hspace*{6pt} | \hspace*{6pt} ( \Phi \lor \Phi ) $
\end{center}
where $ P $ is a \emph{propositional variable}, $ \bot $ is \emph{contradiction}, $ \neg $ is \emph{negation}, $ \to $ is \emph{implication}, $ \land $ is \emph{conjunction}, and $ \lor $ is \emph{disjunction}.
\end{definition}

Given a set $ \Gamma $ of propositions and a proposition $ \varphi $, the relation $ \Gamma \vdash \varphi $ states that there is a derivation with conclusion $ \varphi $ from hypotheses in $ \Gamma $. Here, $ \Gamma $ is also called a \emph{context}. If $ \Gamma $ is empty, we write $ \vdash \varphi $ and say that $ \varphi $ is a theorem.

For notational convenience, we use the following abbreviations:\\
$
\begin{array}{rlll}
\hspace*{12pt} \bullet & \varphi _1 , \varphi _2 , \cdots , \varphi _n & \text{for} & \{ \varphi _1 , \varphi _2 , \cdots , \varphi _n \} ; \\
\bullet & \Gamma , \varphi & \text{for} & \Gamma \cup \{ \varphi \} .
\end{array}
$
\\
\\
The \emph{natural deduction system}, one kind of proof calculi, allows one to derive conclusions from premises. The logical reasoning in this system is expressed by inference rules which are closely related to the ``natural'' way of reasoning. The general form of an \emph{inference rule} is
\begin{prooftree}
\AxiomC{$ P_1 , \cdots , P_n $}
\RightLabel{name of the inference rule}
\UnaryInfC{$ Q $}
\end{prooftree}
where $ P_1 , \cdots , P_n $ are premises and $ Q $ is the conclusion. A rule without premises is an \emph{axiom}. Each inference rule is an atomic step in a derivation which demonstrates how the relation $ \vdash $ is built.

\begin{definition}[\textbf{Natural Deduction System}]
\label{definition:nat_ded_sys}
Given a set of propositional variable, the relation $ \Gamma \vdash \varphi $ is obtained by using the following axiom and inference rules
\begin{myitemize}
\item \emph{Axiom} ($ axiom $)
\begin{prooftree}
\AxiomC{}
\RightLabel{($ axiom $)}
\UnaryInfC{$ \varphi \vdash \varphi $}
\end{prooftree}
Intuitively, one can conclude proposition $ \varphi $ since it already appears in the context. Someone may give it a more general form ``$ \Gamma , \varphi \vdash \varphi $''. However, this form can be obtained by using $ \varphi \vdash \varphi $ and weakening which is given below as another inference rule.

\item \emph{Adding hypotheses into context} ($ add $)
\begin{prooftree}
\AxiomC{$ \Gamma \vdash \varphi $}
\RightLabel{($ add $)}
\UnaryInfC{$ \Gamma , \psi \vdash \varphi $}
\end{prooftree}
The property captured in this rule states that hypotheses of any derived conclusions can be extended with additional assumptions. This rule is also called \emph{weakening}.

\item \emph{$ \to $-introduction} ($ \to $I)
\begin{prooftree}
\AxiomC{$ \Gamma , \varphi \vdash \psi $}
\RightLabel{($ \to $I)}
\UnaryInfC{$ \Gamma \vdash \varphi \to \psi $}
\end{prooftree}
If one can derive $ \psi $ from the context with $ \phi $ as a hypothesis, then $ \varphi \to \psi $ is derivable from the same context without $ \varphi $.

\item \emph{$ \to $-elimination} ($ \to $E)
\begin{prooftree}
\AxiomC{$ \Gamma_1 \vdash \varphi \to \psi $}
\AxiomC{$ \Gamma_2 \vdash \varphi $}
\RightLabel{($ \to $E)}
\BinaryInfC{$ \Gamma_1 \cup \Gamma_2 \vdash \psi $}
\end{prooftree}
If both the conditional claim ``if $ \varphi $ then $ \psi $'' and $ \varphi $ are provided, one can conclude $ \psi $. As mentioned at the beginning, this is a very common inference rule which is also called \emph{modus ponens}.

\item \emph{$ \land $-introduction} ($ \land $I)
\begin{prooftree}
\AxiomC{$ \Gamma \vdash \varphi $}
\AxiomC{$ \Gamma \vdash \psi $}
\RightLabel{($ \land $I)}
\BinaryInfC{$ \Gamma \vdash \varphi \land \psi $}
\end{prooftree}
If both $ \varphi $ and $ \psi $ are derivable from $ \Gamma $, $ \varphi \land \psi $ is also derivable.

\item \emph{$ \land $-elimination} ($ \land $E)
\begin{center}
\AxiomC{$ \Gamma \vdash \varphi \land \psi $}
\RightLabel{($ \land $E$ _1 $)}
\UnaryInfC{$ \Gamma \vdash \varphi $}
\DisplayProof \hspace{10pt}
\AxiomC{$ \Gamma \vdash \varphi \land \psi $}
\RightLabel{($ \land $E$ _2 $)}
\UnaryInfC{$ \Gamma \vdash \psi $}
\DisplayProof
\end{center}
Provided that conjunction $ \varphi \land \psi $ is derivable from $ \Gamma $, both of its components are also derivable.

\item \emph{$ \lor $-introduction} ($ \lor $I)
\begin{center}
\AxiomC{$ \Gamma \vdash \varphi $}
\RightLabel{($ \lor $I$ _1 $)}
\UnaryInfC{$ \Gamma \vdash \varphi \lor \psi $}
\DisplayProof \hspace{10pt}
\AxiomC{$ \Gamma \vdash \psi $}
\RightLabel{($ \lor $I$ _2 $)}
\UnaryInfC{$ \Gamma \vdash \varphi \lor \psi $}
\DisplayProof
\end{center}
One can conclude disjunction $ \varphi \lor \psi $ from either $ \varphi $ or $ \psi $.

\item \emph{$ \lor $-elimination} ($ \lor $E)
\begin{prooftree}
\AxiomC{$ \Gamma \vdash \varphi \lor \psi $}
\AxiomC{$ \Gamma \vdash \varphi \to \rho $}
\AxiomC{$ \Gamma \vdash \psi \to \rho $}
\RightLabel{($ \lor $E)}
\TrinaryInfC{$ \Gamma \vdash \rho $}
\end{prooftree}
If $ \rho $ follows $ \varphi $, $ \rho $ follows $ \psi $ and $ \varphi \lor \psi $, one can conclude $ \rho $.

\item \emph{$ \bot $-elimination} $ \bot $E
\begin{prooftree}
\AxiomC{$ \Gamma \vdash \bot $}
\RightLabel{($ \bot $E)}
\UnaryInfC{$ \Gamma \vdash \varphi $}
\end{prooftree}
From contradiction $ \bot $, we can derive any propositions. This rule is also called \emph{principle of explosion} or \emph{ex falso quodlibet}.

\end{myitemize}
\end{definition}

What syntactically makes it different from classical propositional logic is that \emph{reductio ad absurdum} ($ RAA $) is not a rule in the natural deduction system of intuitionistic logic, which results in some theorems in classical logic being unprovable in intuitionistic logic.

%---------------------------------------------------------------------------%
%Proof
\subsubsection{Proofs}
\label{sec:bg_il_p}
In intuitionistic logic, the judgements about propositions are not based on truth values. We say a proposition in intuitionistic logic holds if we can construct a \emph{proof tree}, also called a \emph{proof}, by using the inference rules in the natural deduction system (definition \ref{definition:nat_ded_sys}).

Some examples are given here to demonstrate how a proof of a proposition is built:
\begin{myitemize}
\item[(1)] $ \varphi , \neg \varphi \vdash \bot $
\begin{prooftree}
\AxiomC{}
\UnaryInfC{$ \varphi \vdash \varphi $}
  \AxiomC{}
  \UnaryInfC{$ \neg \varphi \vdash \neg \varphi $}
  \RightLabel{($ \neg \varphi $ stands for $ \varphi \to \bot $)}
  \UnaryInfC{$ \neg \varphi \vdash \varphi \to \bot $}
\RightLabel{($ \to $E)}
\BinaryInfC{$ \varphi , \neg \varphi \vdash \bot $}
\end{prooftree}
\item[(2)] $ \vdash \varphi \to \neg \neg \varphi $
\begin{prooftree}
\AxiomC{}
\RightLabel{(1)}
\UnaryInfC{$ \varphi , \neg \varphi \vdash \bot $}
\RightLabel{($ \to $I)}
\UnaryInfC{$ \varphi \vdash \neg \varphi \to \bot $}
\RightLabel{($ \neg \neg \varphi $ stands for $ \neg \varphi \to \bot $)}
\UnaryInfC{$ \varphi \vdash \neg \neg \varphi $}
\end{prooftree}
\item[(3)] $\vdash \psi \to ( \varphi \to \psi ) $
\begin{prooftree}
\AxiomC{}
\UnaryInfC{$ \psi \vdash \psi $}
\RightLabel{($ add $)}
\UnaryInfC{$ \psi , \varphi \vdash \psi $}
\RightLabel{($ \to $I)}
\UnaryInfC{$ \psi \vdash \varphi \to \psi $}
\RightLabel{($ \to $I)}
\UnaryInfC{$\vdash \psi \to ( \varphi \to \psi ) $}
\end{prooftree}
\item[(4)] $ \vdash \neg \varphi \to ( \varphi \to \psi ) $
\begin{prooftree}
\AxiomC{}
\RightLabel{(1)}
\UnaryInfC{$ \neg \varphi , \varphi \vdash \bot $}
\RightLabel{($ \bot $E)}
\UnaryInfC{$ \neg \varphi , \varphi \vdash \psi $}
\RightLabel{($ \to $I)}
\UnaryInfC{$ \neg \varphi \vdash \varphi \to \psi $}
\RightLabel{($ \to $I)}
\UnaryInfC{$ \vdash \neg \varphi \to ( \varphi \to \psi ) $}
\end{prooftree}
\item[(5)] $ \vdash ( \neg \varphi \lor \psi ) \to ( \varphi \to \psi ) $
\begin{prooftree}
\AxiomC{}
\UnaryInfC{$ \neg \varphi \lor \psi \vdash \neg \varphi \lor \psi $}
 \AxiomC{(3)}
  \AxiomC{(4)}
\RightLabel{($ \lor $E)}
\TrinaryInfC{$ \neg \varphi \lor \psi \vdash \varphi \to \psi $}
\RightLabel{($ \to $I)}
\UnaryInfC{$ \vdash ( \neg \varphi \lor \psi ) \to ( \varphi \to \psi ) $}
\end{prooftree}
\end{myitemize}
\mbox\\

Though double negation introduction ``$ \varphi \to \neg \neg \varphi $'' is a theorem in intuitionistic propositional logic, double negation elimination ``$ \neg \neg \varphi \to \varphi $'' is unprovable. In classical logic, double negation elimination can be proven by using the rule \emph{reductio ad absurdum}. However, the natural deduction system in intuitionistic propositional logic does not contain $RAA$. That's why double negation elimination cannot be proven in this system.

By the same token, the law of excluded middle ``$ \varphi \lor \neg \varphi $'' is unprovable, either. We can also look at it from an intuitive view. A proof of disjunction $ \varphi \lor \psi $ is either a proof of $ \varphi $ or a proof of $ \psi $. Intuitionists only accept the proof if they know it is exactly the proof of $ \varphi $ or the one of $ \psi $. For an arbitrary proposition $ \varphi $, however, we do not know whether $ \varphi $ has a proof or $ \neg \varphi $ has a proof. Therefore, we cannot give $ \varphi \lor \neg \varphi $ a general proof and it cannot be proven in intuitionistic propositional logic.