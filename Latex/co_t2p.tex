\subsection{Every type-derivation in $ \lambda ^\to $ leads to a proof in intuitionistic implicational logic}
\label{sec:co_t2p}
The proof trees in \ref{sec:bg_il_s} have similar shapes to the type-derivation trees in \ref{sec:bg_lc_stlc}. In fact, the rules in natural deduction system, for constructing relations $ \Gamma \vdash \varphi $, work in the same way as the typing rules, for building well-typed $ \lambda $-terms $ \Gamma \triangleright x: \sigma $. Let us have a look at a simple example first, the relation between type-derivations in simply typed lambda calculus $ \lambda ^\to $ and proofs in intuitionistic implicational logic.

\begin{prooftree}
\AxiomC{}
\UnaryInfC{$ x \tc \sigma \triangleright x: \sigma $}
\RightLabel{($ add $)}
\UnaryInfC{$ x \tc \sigma , y \tc \tau \triangleright x: \sigma $}
\RightLabel{($ \to $I)}
\UnaryInfC{$ x \tc \sigma \triangleright \lambda y \tc \tau . x: \tau \ta \sigma $}
\RightLabel{($ \to $I)}
\UnaryInfC{$ \triangleright \lambda x \tc \sigma . \lambda y \tc \tau . x: \sigma \ta ( \tau \ta \sigma ) $}
\end{prooftree}

The tree above is a type-derivation of $ \lambda $-term $ \lambda x \tc \sigma . \lambda y \tc \tau . x: \sigma \ta ( \tau \ta \sigma ) $. Let us erase the terms in the tree. If we consider every type-symbol as a propositional variable and the relation ``$ \triangleright $'' as ``$ \vdash $'', then the result is exactly a proof tree, which means we obtain a proof of the conclusion represented by the type of the term in the bottom of the type-derivation. Here, the proven proposition is $ \sigma \to ( \tau \to \sigma ) $.
\begin{center}
\AxiomC{}
\UnaryInfC{\sout{$ x \tc $} $\sigma \triangleright$ \sout{$ x: $} $ \sigma $}
\RightLabel{($ add $)}
\UnaryInfC{\sout{$ x \tc $} $ \sigma $, \sout{$ y \tc $} $\tau \triangleright $ \sout{$ x: $} $ \sigma $}
\RightLabel{($ \to $I)}
\UnaryInfC{\sout{$ x \tc $} $ \sigma \triangleright $ \sout{$ \lambda y \tc \tau . x: $} $ \tau \ta \sigma $}
\RightLabel{($ \to $I)}
\UnaryInfC{$ \triangleright $ \sout{$ \lambda x \tc \sigma . \lambda y \tc \tau . x: $} $ \sigma \ta ( \tau \ta \sigma ) $}
\DisplayProof $ \Longrightarrow $
\AxiomC{}
\UnaryInfC{$ \sigma \vdash \sigma $}
\RightLabel{($ add $)}
\UnaryInfC{$ \sigma , \tau \vdash \sigma $}
\RightLabel{($ \to $I)}
\UnaryInfC{$ \sigma \vdash \tau \to \sigma $}
\RightLabel{($ \to $I)}
\UnaryInfC{$ \vdash \sigma \to ( \tau \to \sigma ) $}
\DisplayProof
\end{center}
\mbox{}

\begin{proposition}
\label{proposition:t2p}
Every type-derivation in $ \lambda ^\to $ leads to a proof in intuitionistic implicational logic.
\end{proposition}

\begin{proof}\mbox\\

This proposition can be proven by induction on the height of type-derivation trees. In order to make the proof more readable, the types are denoted by using the propositional variable symbols ($ \varphi , \psi ... $).

The base case is the one when the height of the tree is 1, i.e. a type-derivation without premises, which is the typing rule ($ axiom $). By erasing the term variable and replacing the relation symbol ``$ \triangleright $'' by ``$ \vdash $'', we obtain the axiom in the natural deduction system.
\begin{center}
\AxiomC{}
\RightLabel{($ axiom $)}
\UnaryInfC{$ x \tc \varphi \triangleright x: \varphi $}
\DisplayProof \hspace{10pt} $ \Longrightarrow $ \hspace{10pt}
\AxiomC{}
\RightLabel{($ axiom $)}
\UnaryInfC{$ \varphi \vdash \varphi $}
\DisplayProof
\end{center}

Assume that from every type-derivation of height at most $ n $, we get a corresponding proof by erasing the terms in the tree. Then, what we need to prove is that the tree of height $ n+1 $ also leads to a corresponding proof. According to the typing rules of $ \lambda ^\to $, there are two inductive cases, one for lambda abstraction and another for application:

(1) Lambda abstraction

The induction hypothesis in this case is that the following type-derivation has height $ n $ and leads to a proof (on its right side),
\begin{center}
\AxiomC{$ \vdots $}
\UnaryInfC{$ \Gamma , x \tc \varphi \triangleright M: \psi $}
\DisplayProof \hspace{10pt} $ \Longrightarrow $ \hspace{10pt}
\AxiomC{$ \vdots $}
\UnaryInfC{$ \Gamma ' , \varphi \vdash \psi $}
\DisplayProof
\end{center}
Since the proof tree is obtained by erasing the terms in the derivation tree, the context $ \Gamma ' $ should contain all the types in $ \Gamma $ and nothing else. By using introduction rule for function type on $ \Gamma , x \tc \varphi \triangleright M: \psi $, we have a type-derivation ending with term $ \lambda x \tc \varphi .M$:
\begin{center}
\AxiomC{$ \vdots $}
\UnaryInfC{$ \Gamma , x \tc \varphi \triangleright M: \psi $}
\RightLabel{($ \to $I)}
\UnaryInfC{$ \Gamma \triangleright \lambda x \tc \varphi .M: \varphi \ta \psi $}
\DisplayProof
\end{center}
By using introduction rule for implication on $ \Gamma ' , \varphi \vdash \psi $, we have the following proof:
\begin{center}
\AxiomC{$ \vdots $}
\UnaryInfC{$ \Gamma ' , \varphi \vdash \psi $}
\RightLabel{($ \to $I)}
\UnaryInfC{$ \Gamma ' \vdash \varphi \to \psi $}
\DisplayProof
\end{center}
The conclusion of the proof, $ \varphi \to \psi $, is exactly the type of term $ \lambda x \tc \varphi .M$. Besides, $ \Gamma ' $ contains all the types in $ \Gamma $. As a result, the proof of $ \Gamma ' \vdash \varphi \to \psi $ can be obtained by deleting the terms in the type derivation of  $ \Gamma \triangleright \lambda x \tc \varphi .M: \varphi \to \psi $.\\

(2) Application

Assume that each of the following two derivation-trees, one of which has height $ n $ and another has height at most $ n $, leads to a corresponding proof, and that $ \Gamma_1 \cup \Gamma_2 $ is consistent:
\begin{center}
\AxiomC{$ \vdots $}
\UnaryInfC{$ \Gamma_1 \triangleright M: \varphi \ta \psi $}
\DisplayProof \hspace{10pt} $ \Longrightarrow $ \hspace{10pt}
\AxiomC{$ \vdots $}
\UnaryInfC{$ \Gamma_1 ' \vdash \varphi \to \psi $}
\DisplayProof
\end{center}
\begin{center}
\AxiomC{$ \vdots $}
\UnaryInfC{$ \Gamma_2 \triangleright N: \varphi $}
\DisplayProof \hspace{10pt} $ \Longrightarrow $ \hspace{10pt}
\AxiomC{$ \vdots $}
\UnaryInfC{$ \Gamma_2 ' \vdash \varphi $}
\DisplayProof
\end{center}
Similarly to (1), $ \Gamma_1 ' $ contains all the types in $ \Gamma_1 $ while $ \Gamma_2 ' $ contains all the types in $ \Gamma_2 $. In accordance with function application, we obtain a derivation of term $ MN $:
\begin{center}
\AxiomC{$ \vdots $}
\UnaryInfC{$ \Gamma_1 \triangleright M: \varphi \ta \psi $}
  \AxiomC{$ \vdots $}
  \UnaryInfC{$ \Gamma_2 \triangleright N: \varphi $}
\RightLabel{($ \to $E)}
\BinaryInfC{$ \Gamma_1 \cup \Gamma_2 \triangleright MN: \psi $}
\DisplayProof
\end{center}
We also get a proof of $ \psi $ by applying \emph{modus ponens} on $ \varphi \to \psi $ and $ \varphi $:
\begin{center}
\AxiomC{$ \vdots $}
\UnaryInfC{$ \Gamma_1 ' \vdash \varphi \to \psi $}
  \AxiomC{$ \vdots $}
  \UnaryInfC{$ \Gamma_2 ' \vdash \varphi $}
\RightLabel{($ \to $E)}
\BinaryInfC{$ \Gamma_1 ' \cup \Gamma_2 ' \vdash \psi $}
\DisplayProof
\end{center}
It is obvious that $ \Gamma_1 ' \cup \Gamma_2 ' $ contains all types in $ \Gamma_1 \cup \Gamma_2 $ but nothing else. Hence, the proof of $ \Gamma_1 ' \cup \Gamma_2 ' \vdash \psi $ can be obtained by erasing all terms in the derivation of $ \Gamma_1 \cup \Gamma_2 \triangleright MN: \psi $.\\

Therefore, by erasing all terms in each type-derivation, one can obtain a corresponding proof. In other words, every type-derivation in $ \lambda ^\to $ leads to a proof in intuitionistic implicational logic.

\end{proof}