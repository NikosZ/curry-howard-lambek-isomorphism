\subsection{Every type-derivation in $ \lambda ^\to $ leads to a proof in intuitionistic implicational logic}
\label{sec:co_t2p}
The proof trees in \ref{sec:bg_il_s} have a similar shape to the type-derivation trees in \ref{sec:bg_lc_stlc}. In fact, the rules in natural deduction system to construct relations $ \Gamma \vdash \varphi $ work in the same way as the typing rules to build well-typed $ \lambda $-terms $ \Gamma \triangleright x: \sigma $. Let us have a look at the simplest example first, the type-derivations in simply typed lambda calculus $ \lambda ^\to $ and the proofs in intuitionistic implicational logic.

\begin{prooftree}
\AxiomC{}
\UnaryInfC{$ x: \sigma \triangleright x: \sigma $}
\RightLabel{($ add $)}
\UnaryInfC{$ x: \sigma , y: \tau \triangleright x: \sigma $}
\RightLabel{($ \to $I)}
\UnaryInfC{$ x: \sigma \triangleright \lambda y: \tau . x: \tau \to \sigma $}
\RightLabel{($ \to $I)}
\UnaryInfC{$ \emptyset \triangleright \lambda x: \sigma . \lambda y: \tau . x: \sigma \to ( \tau \to \sigma ) $}
\end{prooftree}

The tree above is a type-derivation of $ \lambda $-term $ \lambda x: \sigma . \lambda y: \tau . x: \sigma \to ( \tau \to \sigma ) $. Let us erase the terms in the tree. If we consider every type-symbol as a propositional variable and the relation ``$ \triangleright $'' as ``$ \vdash $'', then the tree in the right side is exactly a proof tree, which means we obtain a proof of the conclusion represented by the type of the term in the bottom of the tree. Here, the proven proposition is $ \sigma \to ( \tau \to \sigma ) $.
\begin{center}
\AxiomC{}
\UnaryInfC{\sout{$ x: $} $\sigma \triangleright$ \sout{$ x: $} $ \sigma $}
\RightLabel{($ add $)}
\UnaryInfC{\sout{$ x: $} $ \sigma $, \sout{$ y: $} $\tau \triangleright $ \sout{$ x: $} $ \sigma $}
\RightLabel{($ \to $I)}
\UnaryInfC{\sout{$ x: $} $ \sigma \triangleright $ \sout{$ \lambda y: \tau . x: $} $ \tau \to \sigma $}
\RightLabel{($ \to $I)}
\UnaryInfC{$ \emptyset \triangleright $ \sout{$ \lambda x: \sigma . \lambda y: \tau . x: $} $ \sigma \to ( \tau \to \sigma ) $}
\DisplayProof $ \Longrightarrow $
\AxiomC{}
\UnaryInfC{$ \sigma \triangleright \sigma $}
\RightLabel{($ add $)}
\UnaryInfC{$ \sigma , \tau \triangleright \sigma $}
\RightLabel{($ \to $I)}
\UnaryInfC{$ \sigma \triangleright \tau \to \sigma $}
\RightLabel{($ \to $I)}
\UnaryInfC{$ \emptyset \triangleright \sigma \to ( \tau \to \sigma ) $}
\DisplayProof
\end{center}
\mbox{}

\begin{proposition}
\label{proposition:t2p}
Every type-derivation in $ \lambda ^\to $ leads to a proof in intuitionistic implicational logic.
\end{proposition}

\begin{proof}
This proposition can be proven by induction on the height of type-derivation trees. In order to make the proof more readable, the types are denoted by using the propositional variable symbols ($ \varphi , \psi ... $).

The base case is the one when the height of the tree is 1, i.e. a type-derivation without hypotheses which is the typing rule ($ axiom $). By erasing the term and replacing the relation symbol ``$ \triangleright $'' by ``$ \vdash $'', we obtain the axiom in the natural deduction system (the right tree).
\begin{center}
\AxiomC{}
\RightLabel{($ axiom $)}
\UnaryInfC{$ x: \varphi \triangleright x: \varphi $}
\DisplayProof \hspace{10pt} $ \Longrightarrow $ \hspace{10pt}
\AxiomC{}
\RightLabel{($ axiom $)}
\UnaryInfC{$ \varphi \vdash \varphi $}
\DisplayProof
\end{center}

Suppose for very type-derivation tree of height at most $ n $, we get a corresponding proof by erasing the terms in the tree. Then, we want to show that the tree of height $ n+1 $ also leads to a corresponding proof. According to the typing rules of $ \lambda ^\to $, there are two inductive cases, one with lambda abstraction and another with application:
\begin{myitemize}
\item[(1)] Lambda abstraction\\
Assume that the following type-derivation of height $ n $ leads to a proof on its right side,
\begin{center}
\AxiomC{$ \vdots $}
\UnaryInfC{$ \Gamma , x: \varphi \triangleright M: \psi $}
\DisplayProof \hspace{10pt} $ \Longrightarrow $ \hspace{10pt}
\AxiomC{$ \vdots $}
\UnaryInfC{$ \Gamma ' , \varphi \vdash \psi $}
\DisplayProof
\end{center}
Since the proof tree is obtained by erasing the terms in the derivation tree, the context $ \Gamma ' $ should contain all the types in $ \Gamma $ but nothing else. By using introduction rule for function type on $ \Gamma , x: \varphi \triangleright M: \psi $, we have a type-derivation of term $ \lambda x: \varphi .M$
\begin{center}
\AxiomC{$ \vdots $}
\UnaryInfC{$ \Gamma , x: \varphi \triangleright M: \psi $}
\RightLabel{($ \to $I)}
\UnaryInfC{$ \Gamma \triangleright \lambda x: \varphi .M: \varphi \to \psi $}
\DisplayProof
\end{center}
By using introduction rule for implication on $ \Gamma ' , \varphi \vdash \psi $, we have the following proof
\begin{center}
\AxiomC{$ \vdots $}
\UnaryInfC{$ \Gamma ' , \varphi \vdash \psi $}
\RightLabel{($ \to $I)}
\UnaryInfC{$ \Gamma ' \vdash \varphi \to \psi $}
\DisplayProof
\end{center}
It is obvious that $ \Gamma ' \vdash \varphi \to \psi $ can also be obtained by deleting the terms in $ \Gamma $ which is $ \Gamma ' $ and the term $ \lambda x: \varphi .M $. Hence, the derivation of $ \Gamma \triangleright \lambda x: \varphi .M: \varphi \to \psi $ leads to a proof of $ \Gamma ' \vdash \varphi \to \psi $.
\item[(2)] Application\\
Assume the following two derivation-trees, one of which has height $ n $ and another has height at most $ n $, lead to two proofs
\begin{center}
\AxiomC{$ \vdots $}
\UnaryInfC{$ \Gamma \triangleright M: \varphi \to \psi $}
\DisplayProof \hspace{10pt} $ \Longrightarrow $ \hspace{10pt}
\AxiomC{$ \vdots $}
\UnaryInfC{$ \Gamma ' \vdash \varphi \to \psi $}
\DisplayProof
\end{center}
\begin{center}
\AxiomC{$ \vdots $}
\UnaryInfC{$ \Gamma \triangleright N: \varphi $}
\DisplayProof \hspace{10pt} $ \Longrightarrow $ \hspace{10pt}
\AxiomC{$ \vdots $}
\UnaryInfC{$ \Gamma ' \vdash \varphi $}
\DisplayProof
\end{center}
Similarly, $ \Gamma ' $ contains all the types in $ \Gamma $. Then, according to function application, we obtain a derivation of $ MN$
\begin{center}
\AxiomC{$ \vdots $}
\UnaryInfC{$ \Gamma \triangleright M: \varphi \to \psi $}
  \AxiomC{$ \vdots $}
  \UnaryInfC{$ \Gamma \triangleright N: \varphi $}
\RightLabel{($ \to $E)}
\BinaryInfC{$ \Gamma \triangleright MN: \psi $}
\DisplayProof
\end{center}
We also get a proof of $ \psi $ by applying \emph{modus ponens} on $ \varphi \to \psi $ and $ \varphi $
\begin{center}
\AxiomC{$ \vdots $}
\UnaryInfC{$ \Gamma ' \vdash \varphi \to \psi $}
  \AxiomC{$ \vdots $}
  \UnaryInfC{$ \Gamma ' \vdash \varphi $}
\RightLabel{($ \to $E)}
\BinaryInfC{$ \Gamma ' \vdash \psi $}
\DisplayProof
\end{center}
Then we can see that if we erase the terms in $ \Gamma $ and term $ MN $, we also have $ \Gamma ' \vdash \psi $. Hence, the derivation of $ \Gamma \triangleright MN: \psi $ leads to a proof of $ \Gamma ' \vdash \psi $.
\end{myitemize}
Therefore, every type-derivation in $ \lambda ^\to $ leads to a proof in intuitionistic implicational logic.
\end{proof}