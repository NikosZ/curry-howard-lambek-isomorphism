\section{Reflections}
Doing a dissertation-only project was the most challenging but ambitious decision that I have made during my life in the University of Birmingham. As an overseas student, writing (in English) is definitely the most painful part of this project.

Though the correspondence seems to be straightforward, one needs to be very careful in the proofs. As far as some details are concerned, some terms may not correspond to proofs very tightly. The following is such a counter example.
\begin{center}
\AxiomC{}
\UnaryInfC{$ x \tc \sigma \vdash x : \sigma $}
\RightLabel{($ add $)}
\UnaryInfC{$ x \tc \sigma , y \tc \sigma \vdash x : \sigma $}
\RightLabel{($ \to $I)}
\UnaryInfC{$ x \tc \sigma \vdash \lambda y \tc \sigma. x : \sigma \ta \sigma $}
\RightLabel{($ \to $I)}
\UnaryInfC{$ \vdash \lambda x \tc \sigma. \lambda y \tc \sigma. x : \sigma \ta (\sigma \ta \sigma) $}
\DisplayProof \hspace*{10pt} $ \Longrightarrow $ \hspace*{10pt}
\AxiomC{}
\UnaryInfC{$ \sigma \vdash \sigma $}
\RightLabel{($ add $)}
\UnaryInfC{$ \sigma , \sigma \vdash \sigma $}
\RightLabel{($ \to $I)}
\UnaryInfC{$ \sigma \vdash \sigma \to \sigma $}
\RightLabel{($ \to $I)}
\UnaryInfC{$ \vdash \sigma \to (\sigma \to \sigma) $}
\DisplayProof
\end{center}
We erase all the terms in the derivation of $ \lambda x \tc \sigma. \lambda y \tc \sigma. x : \sigma \ta (\sigma \ta \sigma) $. But the result does not look like a valid proof since the context, as a set, should not contain repeated propositions. One may want to adjust it to obtain a proof of ``$\sigma \to (\sigma \to \sigma)$''. However, we cannot rebuild the original term (or a term $ \alpha $-equivalent to the original one) from this proof.
\begin{center}
\AxiomC{}
\UnaryInfC{$ \sigma \vdash \sigma $}
\RightLabel{($ \to $I)}
\UnaryInfC{$ \vdash \sigma \to \sigma $}
\RightLabel{($ add $)}
\UnaryInfC{$ \sigma \vdash \sigma \to \sigma $}
\RightLabel{($ \to $I)}
\UnaryInfC{$ \vdash \sigma \to (\sigma \to \sigma) $}
\DisplayProof \hspace*{10pt} $ \Longrightarrow $ \hspace*{10pt}
\AxiomC{}
\UnaryInfC{$ x \tc \sigma \vdash x: \sigma $}
\RightLabel{($ \to $I)}
\UnaryInfC{$ \vdash \lambda x \tc \sigma . x : \sigma \ta \sigma $}
\RightLabel{($ add $)}
\UnaryInfC{$ y \tc \sigma \vdash \lambda x \tc \sigma . x : \sigma \ta \sigma $}
\RightLabel{($ \to $I)}
\UnaryInfC{$ \vdash \lambda y \tc \sigma . \lambda x \tc \sigma . x : \sigma \ta (\sigma \ta \sigma) $}
\DisplayProof
\end{center}
One of the solutions to this problem is linear logic which is proposed as a refinement of classical and intuitionistic logic. But linear logic is not involved in this project.

During this project, most of the time was spent in learning foundations of category theory. It was difficult to show that \textsc{Pos}$_{\bot !}$ (the category of posets with bottom preserving maps given in section \ref{sec:bg_cat_ccc}) is not cartesian closed. Even though the supervisor allowed me to leave this tough question, I kept trying to solve it for several weeks. However, I was just able to show that the exponentials from \textsc{Pos} are not categorical exponentials in \textsc{Pos}$_{\bot !}$. Finally, this question was overcome under the guidance of the supervisor.

The literature \cite{AL91,BW95,LS86,Mit96} I have read, interprets $ \lambda^{unit,\times,\to} $ terms in cartesian closed categories and shows the cartesian closed structure of the category generated from $ \lambda^{unit,\times,\to} $. As discussed at the beginning of section \ref{sec:co_t2m}, their connection seems to be obvious. People are much more interested in revealing the obscure connection; therefore, with the advice of the supervisor, it was decided to prove the correspondence between $ \lambda^{\to} $ and CCCs, which is the main part of this project as well as my own contribution to this project.

Cartesian closed categories have several properties of importance, i.e. some special equations which hold in every CCC. Most categorical literature utilises diagrams to give proofs to these equations. However, beginners may find it difficult to see equations through diagrams. This dissertation provides both diagram-based and equation-based proofs to the equations. Both ways have their pros and cons. Given a complicated equation, its corresponding diagram may be too large to draw on an A4 paper. The equational ones provide proof step by step, which always makes the proof tedious and hard to remember. What is a good way to present proofs? This is another tough question I faced in this project.

All in all, it is really enjoyable to be immersed in this research-oriented project. For one thing, it satisfied my hunger for knowledge in theoretical computer science as well as improved my skills in research. For another, the connection between proof theory, type theory and category theory did amaze me. When looking at different mathematical subjects, I keep asking myself whether they can be linked to each other.